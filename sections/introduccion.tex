\newpage
~
\newpage
\section{Introducción}

\subsection{Contexto}
Hoy en día, en la era de la digitalización, todos tenemos móviles con cámaras de alta resolución y grandes almacenamientos que nos permiten capturar y guardar una gran cantidad de fotografías y vídeos, espacio el cual, aunque parezca enorme, todos sabemos que se acaba terminando. Para ello es que se ofrecen distintos tipos de servicios en la nube que nos permiten tener todos esos archivos multimedia guardados en un lugar desconocido para el usuario promedio junto con la facilidad de la sincronización automática.

Aún así, este espacio también se acaba, junto con la desventaja de que hay que pagar por guardar esos archivos en ese lugar desconocido. Es por eso que este proyecto busca ofrecer una solución a este problema, permitiendo alojar un servidor de almacenamiento y sincronización en cualquier servidor o computadora personal.

\subsection{Motivación}
Durante un periodo vacacional, se presentó un caso práctico en el que un familiar enfrentaba dificultades debido a que su almacenamiento en Google Fotos se había acabado. Este problema planteó la necesidad de implementar un sistema que permitiera la transferencia automática de fotografías desde un dispositivo móvil a un portátil antiguo, aprovechando la conectividad de la red wifi doméstica.

Es por ello que me decidí a solucionar esta problemática mediante el diseño e implementación de una solución tecnológica adecuada, eficiente y segura que permita a cualquier usuario tener un sistema de sincronización de multimedia entre dispositivos, teniendo un servidor central en el que guardar las fotos.

\subsection{Objetivos}
En infinitivo y concisos. Siguiendo las siglas SMART (Specific, Measurable, Achievable, Relevant, Time-bound).
Mejor tener objetivos generales y después específicos.

Mover después de Motivación.
Para definir los objetivos de nuestro proyecto, primero vamos a definir unos objetivos generales de los cuales se derivarán los específicos.
\begin{itemize}
    \item OG1: Desarrollar un sistema multiplataforma y multiusuario para la compartición de archivos multimedia.
        \begin{itemize}
            \item OE1: Implementar un cliente de sincronización de archivos multimedia.
            \item OE2: Implementar un servidor de sincronización de archivos multimedia.
            \item OE3: Implementar un sistema de gestión de usuarios y permisos.
            \item OE4: Implementar un sistema de gestión de archivos multimedia.
        \end{itemize}
    \item OG2: Realizar un proyecto Open-Source
        \begin{itemize}
            \item OE5: Publicar de manera pública el código fuente del proyecto.
            \item OE6: Documentar todos y cada uno de los componentes del proyecto.
            \item OE7: Utilizar una arquitectura de software que facilite la comprensión y el mantenimiento del código.
        \end{itemize}
    \item OG3: Proporcionar una solución segura y eficiente para la sincronización de archivos multimedia.
        \begin{itemize}
            \item OE8: Implementar un sistema de cifrado de archivos multimedia.
            \item OE9: Implementar un sistema de autenticación y autorización de usuarios.
        \end{itemize}
    \item OG4: Desarrollar un sistema fácilmente escalable
        \begin{itemize}
            \item OE10: Implementar un sistema poco acoplado que nos permita fácil escalabilidad.
            \item OE11: Utilizar soluciones software que nos permitan escalar el sistema de manera horizontal.
        \end{itemize}
    \item OG5: Proporcionar una solución que permita realizar copias de seguridad de una manera sencilla.
    \item OG6: Desarrollar una solución nativa para las aplicaciones móviles más utilizadas.
        \begin{itemize}
            \item OE12: Conseguir un rendimiento óptimo en móviles.
            \item OE13: Implementar una interfaz de usuario intuitiva y fácil de usar.
            \item OE14: Conseguir una buena experiencia de usuario.
        \end{itemize}

\end{itemize}
