\section{Introducción}

\subsection{Contexto}
Hoy en día, en la era de la digitalización, todos tenemos móviles con cámaras de alta resolución y grandes almacenamientos que nos permiten capturar y guardar una gran cantidad de fotografías y vídeos, espacio el cual, aunque parezca enorme, todos sabemos que se acaba terminando. Para ello es que se ofrecen distintos tipos de servicios en la nube que nos permiten tener todos esos archivos multimedia guardados en un lugar desconocido para el usuario promedio junto con la facilidad de la sincronización automática.

Aún así, este espacio también se acaba, junto con la desventaja de que hay que pagar por guardar esos archivos en ese lugar desconocido. Es por eso que este proyecto busca ofrecer una solución a este problema, permitiendo alojar un servidor de almacenamiento y sincronización en cualquier servidor o computadora personal.

\subsection{Motivación}
Durante un periodo vacacional, se presentó un caso práctico en el que un familiar enfrentaba dificultades debido a que su almacenamiento en Google Fotos se había acabado. Este problema planteó la necesidad de implementar un sistema que permitiera la transferencia automática de fotografías desde un dispositivo móvil a un portátil antiguo, aprovechando la conectividad de la red wifi doméstica.

Es por ello que me decidí a solucionar esta problemática mediante el diseño e implementación de una solución tecnológica adecuada, eficiente y segura que permita a cualquier usuario tener un sistema de sincronización de multimedia entre dispositivos, teniendo un servidor central en el que guardar las fotos.
