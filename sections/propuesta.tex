\section{Propuesta}
Ya que es un tfg de desarrollo describimos lo realizado y cómo, así como los resultados obtenidos.

\subsection{Metodología}
Para el desarrollo de este proyecto se va a hacer uso de la metodología ágil Scrum. Esta metodología se basa en el desarrollo iterativo e incremental, lo que permite una mayor flexibilidad y adaptación a los cambios durante el proceso de desarrollo.

Separaremos el desarrollo en distintos sprints, cada uno de ellos con una duración de dos semanas.

Durante cada sprint se seleccionarán las historias de usuario\footnote{Las historias de usuario son descripciones concisas y sencillas de una funcionalidad, escritas desde la perspectiva del usuario} al principio del sprint, se realizarán las tareas necesarias para completarlas y al final del sprint se realizará una revisión y una retrospectiva del mismo, donde se evaluará lo que se ha hecho, cómo se puede mejorar y se planificará el siguiente sprint dependiendo del estado del recién terminado.

Gracias a esta metodología conseguimos tener una organización muy clara de lo que se va a hacer, cómo se va a hacer y cuándo se va a hacer.

\subsection{Tecnologías}

\subsection{Historias de usuario}
En esta sección se detallan las historias de usuario de la aplicación, separadas en dos grupos: las de la aplicación de servidor y las de móvil.

Se ha considerado esta separación ya que la aplicación de servidor tiene un objetivo diferente al de la aplicación móvil, de esta manera conseguimos una mejor organización de las historias de usuario.

Durante los primeros sprints se trabajará de manera principalmente separada, enfocándose en la parte correspondiente que se defina de la aplicación y en una fase más avanzada se trabajará de manera conjunta, integrando ambas aplicaciones.

Para la planificación del desarrollo se han utilizado puntos de historia, los cuales representan una estimación de lo que se considera que se tardará en implementar las historias de usuario. Esta estimación es relativa, es decir, no representa un tiempo real sino una estimación con respecto a todas las demás historias de usuario, siendo 1 punto de historia la historia de usuario más sencilla de implementar o que menos tiempo requiere.

Este es un listado inicial de historias de usuario, durante los sprints se irá especificando si alguna historia de usuario ha cambiado, añadido o eliminado del product backlog\footnote{El product backlog es una lista priorizada de requisitos o tareas pendientes en un proyecto ágil.}.

Cada historia de usuario tiene un identificador único, una descripción de la historia de usuario y una estimación en puntos de historia. Ésta es después desglosada en historias de usuario más pequeñas de las cuales se definen tareas que tienen que ser realizadas para completar la historia de usuario con su estimación en horas.

\subsubsection{Servidor}
\renewcommand{\arraystretch}{1.3} % Increases row height for readability
\rowcolors{2}{gray!15}{white} % Alternate row colors

\begin{longtable}{|p{2cm}|p{4cm}|p{6cm}|}
    \hline
    \textbf{ID} & \textbf{Title} & \textbf{Description} \\ \hline
    US1 & Upload Photos & As a user, I want to upload photos to the server so that I can back them up. \\ \hline
    US2 & Sync Files & As a user, I want to sync my files automatically so that I don’t lose data. \\ \hline
    US3 & Manage Users & As an admin, I want to manage users so that I can control access. \\ \hline
\end{longtable}

\subsubsection{Móvil} 

\subsection{Planificación inicial}

\subsection{Presupuesto}

