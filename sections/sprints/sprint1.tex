\subsection{Sprint 1}
Tal y como se ha definido en la sección \ref{sec:planificacion-inicial}, el primer sprint se centra en la base de nuestra aplicación web, desarrollando la estructura inicial del servidor, gestión de usuarios y subida y descarga sencilla de archivos.
Para ello, se han elegido las historias de usuario relacionadas con el objetivo de este sprint y se han desarrollado definiendo sub-historias de usuario si fueran necesarias, criterios de aceptación y las tareas necesarias para su implementación.

Las historias de usuario se han seleccionado de manera aproximada, puesto que al ser el primer sprint no tenemos una velocidad de equipo definida. Existe la posibilidad de que algunas historias de usuario no se completen en este sprint o de que se completen más de las previstas, por lo que se ha dejado un margen de maniobra para que el equipo pueda adaptarse a la realidad del desarrollo.

Cuando se termine el sprint, se calculará la velocidad del equipo la cual se utilizará para planificar los siguientes sprints, de manera que se pueda ajustar la cantidad de historias de usuario a desarrollar en cada uno de ellos.

Una vez definidas todas las tareas que se van a realizar en este sprint, se ha realizado un diagrama de Gantt para planificar el tiempo que se va a dedicar a cada una de ellas, teniendo en cuenta que el sprint tiene una duración de dos semanas, y se dará un orden de prioridad a las tareas que se consideren más importantes para llegar a el objetivo del sprint.

\subsubsection{Historias de usuario}
En esta sección se detallan las historias de usuario y técnicas que se han elegido para este sprint. Se va a hacer una breve descripción de cada una de ellas, así como los criterios de aceptación y las tareas necesarias para su implementación. Si fuera necesario, se definirán sub-historias de usuario para facilitar su desarrollo.

Las seleccionadas son las siguientes:
\begin{itemize}
    \item HU05: Inicio de sesión - 5PH
    \item HU06: Cerrar sesión - 2PH
    \item HU13: Crear cuentas - 8PH
    \item HT03 API REST en Rust - 13PH
    \item HT05: Autenticación JWT - 5PH
    \item HT08: Base de datos - 8PH
    \item HT18.1: Binario - 2PH
    \item HT19: Dockerización - 3PH
    \item HT20.1: Documentación del proyecto en GitHub - 5PH
    \item HT20.2: Documentación de la API REST con OpenAPI - 5PH
\end{itemize}

Estas historias de usuario suman un total de 56 puntos de historia (PH)

\paragraph{Descomposición en tareas de desarrollo}

% HU05: Inicio de sesión
\begin{table}[H]
    \begin{center}
        \begin{tabularx}{\textwidth}{|l|X|l|}
            \hline
            \textbf{Identificador HU05} & 
            \textbf{Como usuario, quiero iniciar sesión con contraseña o clave, para evitar que otros accedan a mis archivos} &
            \textbf{Estimación: 5}\\
            \hline
            \multicolumn{3}{|c|}{
                \begin{minipage}{\linewidth}
                    \centering
                    \vspace{0.5em}
                    \begin{tabular}{|l|p{8cm}|r|}
                        \hline
                        \textbf{Identificador} & \textbf{Título de la tarea de desarrollo} & \makecell{\textbf{Estimación}\\\textbf{(h)}} \\
                        \hline
                        HU05-1 & Implementar endpoint para iniciar sesión & 1 \\
                        \hline
                        HU05-2 & Validar credenciales de usuario contra la base de datos & 1 \\
                        \hline
                        HU05-3 & Generar y devolver token JWT al usuario autenticado & 1 \\
                        \hline
                        HU05-4 & Gestionar errores de autenticación (usuario no existe, contraseña incorrecta) & 1 \\
                        \hline
                        HU05-5 & Documentar el endpoint de inicio de sesión en OpenAPI & 1 \\
                        \hline
                    \end{tabular}
                    \vspace{0.5em}
                \end{minipage}
            } \\
            \hline
            \multicolumn{3}{|p{\textwidth}|}{
                \textbf{Pruebas de aceptación:}
                    \begin{itemize}
                        \item El usuario puede iniciar sesión con su contraseña.
                        \item Cuando el usuario inicia sesión, se le devuelve un token JWT.
                        \item Si el usuario no existe, se devuelve un mensaje de error.
                        \item Si la contraseña o clave son incorrectas, se devuelve un mensaje de error.
                        \item El endpoint está documentado en OpenAPI.
                    \end{itemize}
            }\\
            \hline
            \multicolumn{3}{|p{\textwidth}|}{
                \textbf{Observaciones:}
                \begin{itemize}
                    \item El endpoint debe ser seguro (usar HTTPS).
                    \item El token JWT debe tener expiración configurable.
                \end{itemize}
            }\\
            \hline
        \end{tabularx}
    \end{center}
\end{table}

% HU06: Cerrar sesión
\begin{table}[H]
    \begin{center}
        \begin{tabularx}{\textwidth}{|l|X|l|}
            \hline
            \textbf{Identificador HU06} & 
            \textbf{Como usuario, quiero poder cerrar sesión en un dispositivo, para proteger mis datos si pierdo el móvil} &
            \textbf{Estimación: 2}\\
            \hline
            \multicolumn{3}{|c|}{
                \begin{minipage}{\linewidth}
                    \centering
                    \vspace{0.5em}
                    \begin{tabular}{|l|p{8cm}|r|}
                        \hline
                        \textbf{Identificador} & \textbf{Título de la tarea de desarrollo} & \makecell{\textbf{Estimación}\\\textbf{(h)}} \\
                        \hline
                        HU06-1 & Implementar endpoint para cerrar sesión (invalidar token JWT) & 1 \\
                        \hline
                        HU06-2 & Gestionar lista negra de tokens JWT (opcional) & 1 \\
                        \hline
                    \end{tabular}
                    \vspace{0.5em}
                \end{minipage}
            } \\
            \hline
            \multicolumn{3}{|p{\textwidth}|}{
                \textbf{Pruebas de aceptación:}
                    \begin{itemize}
                        \item El usuario puede cerrar sesión y su token queda invalidado.
                        \item Tras cerrar sesión, el token no permite acceder a endpoints protegidos.
                    \end{itemize}
            }\\
            \hline
            \multicolumn{3}{|p{\textwidth}|}{
                \textbf{Observaciones:}
                \begin{itemize}
                    \item Si no se implementa lista negra, el token expira por tiempo.
                \end{itemize}
            }\\
            \hline
        \end{tabularx}
    \end{center}
\end{table}

% HU13: Crear cuentas
\begin{table}[H]
    \begin{center}
        \begin{tabularx}{\textwidth}{|l|X|l|}
            \hline
            \textbf{Identificador HU13} & 
            \textbf{Como administrador, quiero crear cuentas de usuario con permisos, para que varias personas puedan usar el servidor} &
            \textbf{Estimación: 8}\\
            \hline
            \multicolumn{3}{|c|}{
                \begin{minipage}{\linewidth}
                    \centering
                    \vspace{0.5em}
                    \begin{tabular}{|l|p{8cm}|r|}
                        \hline
                        \textbf{Identificador} & \textbf{Título de la tarea de desarrollo} & \makecell{\textbf{Estimación}\\\textbf{(h)}} \\
                        \hline
                        HU13-1 & Implementar endpoint para crear cuentas de usuario & 2 \\
                        \hline
                        HU13-2 & Validar permisos de administrador para crear cuentas & 1 \\
                        \hline
                        HU13-3 & Añadir roles/permisos a los usuarios & 2 \\
                        \hline
                        HU13-4 & Gestionar almacenamiento seguro de contraseñas (hash) & 1 \\
                        \hline
                        HU13-5 & Documentar el endpoint de creación de cuentas en OpenAPI & 1 \\
                        \hline
                        HU13-6 & Pruebas unitarias de creación de usuario & 1 \\
                        \hline
                    \end{tabular}
                    \vspace{0.5em}
                \end{minipage}
            } \\
            \hline
            \multicolumn{3}{|p{\textwidth}|}{
                \textbf{Pruebas de aceptación:}
                    \begin{itemize}
                        \item Solo el administrador puede crear cuentas.
                        \item El usuario creado puede iniciar sesión.
                        \item Los roles/permisos se asignan correctamente.
                        \item El endpoint está documentado en OpenAPI.
                    \end{itemize}
            }\\
            \hline
            \multicolumn{3}{|p{\textwidth}|}{
                \textbf{Observaciones:}
                \begin{itemize}
                    \item Usar hash seguro para contraseñas (ej: Argon2).
                \end{itemize}
            }\\
            \hline
        \end{tabularx}
    \end{center}
\end{table}

% HT03: API REST en Rust
\begin{table}[H]
    \begin{center}
        \begin{tabularx}{\textwidth}{|l|X|l|}
            \hline
            \textbf{Identificador HT03} & 
            \textbf{Desarrollar API RESTful usando Rust y Axum} &
            \textbf{Estimación: 13}\\
            \hline
            \multicolumn{3}{|c|}{
                \begin{minipage}{\linewidth}
                    \centering
                    \vspace{0.5em}
                    \begin{tabular}{|l|p{8cm}|r|}
                        \hline
                        \textbf{Identificador} & \textbf{Título de la tarea de desarrollo} & \makecell{\textbf{Estimación}\\\textbf{(h)}} \\
                        \hline
                        HT03-1 & Crear estructura base del proyecto en Rust & 2 \\
                        \hline
                        HT03-2 & Configurar Axum y dependencias principales & 2 \\
                        \hline
                        HT03-3 & Definir rutas y controladores básicos & 2 \\
                        \hline
                        HT03-4 & Implementar manejo de errores global & 2 \\
                        \hline
                        HT03-5 & Añadir middlewares (logging, CORS, etc.) & 2 \\
                        \hline
                        HT03-6 & Configurar variables de entorno y settings & 1 \\
                        \hline
                        HT03-7 & Documentar endpoints iniciales & 1 \\
                        \hline
                        HT03-8 & Pruebas de integración básicas & 1 \\
                        \hline
                    \end{tabular}
                    \vspace{0.5em}
                \end{minipage}
            } \\
            \hline
            \multicolumn{3}{|p{\textwidth}|}{
                \textbf{Pruebas de aceptación:}
                    \begin{itemize}
                        \item El servidor arranca y responde a peticiones básicas.
                        \item Los endpoints definidos funcionan correctamente.
                        \item El manejo de errores es consistente.
                    \end{itemize}
            }\\
            \hline
            \multicolumn{3}{|p{\textwidth}|}{
                \textbf{Observaciones:}
                \begin{itemize}
                    \item Seguir estructura modular y buenas prácticas de Rust.
                \end{itemize}
            }\\
            \hline
        \end{tabularx}
    \end{center}
\end{table}

% HT05: Autenticación JWT
\begin{table}[H]
    \begin{center}
        \begin{tabularx}{\textwidth}{|l|X|l|}
            \hline
            \textbf{Identificador HT05} & 
            \textbf{Implementar autenticación con JSON Web Tokens} &
            \textbf{Estimación: 5}\\
            \hline
            \multicolumn{3}{|c|}{
                \begin{minipage}{\linewidth}
                    \centering
                    \vspace{0.5em}
                    \begin{tabular}{|l|p{8cm}|r|}
                        \hline
                        \textbf{Identificador} & \textbf{Título de la tarea de desarrollo} & \makecell{\textbf{Estimación}\\\textbf{(h)}} \\
                        \hline
                        HT05-1 & Añadir librería de JWT y configuración & 1 \\
                        \hline
                        HT05-2 & Implementar generación y validación de tokens & 2 \\
                        \hline
                        HT05-3 & Proteger endpoints con autenticación JWT & 1 \\
                        \hline
                        HT05-4 & Pruebas unitarias de autenticación & 1 \\
                        \hline
                    \end{tabular}
                    \vspace{0.5em}
                \end{minipage}
            } \\
            \hline
            \multicolumn{3}{|p{\textwidth}|}{
                \textbf{Pruebas de aceptación:}
                    \begin{itemize}
                        \item Solo usuarios autenticados pueden acceder a endpoints protegidos.
                        \item Los tokens inválidos o expirados son rechazados.
                    \end{itemize}
            }\\
            \hline
            \multicolumn{3}{|p{\textwidth}|}{
                \textbf{Observaciones:}
                \begin{itemize}
                    \item Usar claves seguras y expiración adecuada.
                \end{itemize}
            }\\
            \hline
        \end{tabularx}
    \end{center}
\end{table}

% HT08: Base de datos
\begin{table}[H]
    \begin{center}
        \begin{tabularx}{\textwidth}{|l|X|l|}
            \hline
            \textbf{Identificador HT08} & 
            \textbf{Implementar SQLite o PostgreSQL para usuarios y archivos} &
            \textbf{Estimación: 8}\\
            \hline
            \multicolumn{3}{|c|}{
                \begin{minipage}{\linewidth}
                    \centering
                    \vspace{0.5em}
                    \begin{tabular}{|l|p{8cm}|r|}
                        \hline
                        \textbf{Identificador} & \textbf{Título de la tarea de desarrollo} & \makecell{\textbf{Estimación}\\\textbf{(h)}} \\
                        \hline
                        HT08-1 & Definir modelo de datos para usuarios y archivos & 2 \\
                        \hline
                        HT08-2 & Crear migraciones iniciales de la base de datos & 2 \\
                        \hline
                        HT08-3 & Implementar acceso a base de datos en Rust & 2 \\
                        \hline
                        HT08-4 & Pruebas de persistencia y consultas básicas & 2 \\
                        \hline
                    \end{tabular}
                    \vspace{0.5em}
                \end{minipage}
            } \\
            \hline
            \multicolumn{3}{|p{\textwidth}|}{
                \textbf{Pruebas de aceptación:}
                    \begin{itemize}
                        \item Se pueden crear, consultar y modificar usuarios y archivos.
                        \item Las migraciones funcionan correctamente.
                    \end{itemize}
            }\\
            \hline
            \multicolumn{3}{|p{\textwidth}|}{
                \textbf{Observaciones:}
                \begin{itemize}
                    \item Usar ORM recomendado para Rust (ej: sqlx, diesel).
                \end{itemize}
            }\\
            \hline
        \end{tabularx}
    \end{center}
\end{table}

% HT18.1: Binario
\begin{table}[H]
    \begin{center}
        \begin{tabularx}{\textwidth}{|l|X|l|}
            \hline
            \textbf{Identificador HT18.1} & 
            \textbf{Empaquetar la aplicación como un solo binario} &
            \textbf{Estimación: 2}\\
            \hline
            \multicolumn{3}{|c|}{
                \begin{minipage}{\linewidth}
                    \centering
                    \vspace{0.5em}
                    \begin{tabular}{|l|p{8cm}|r|}
                        \hline
                        \textbf{Identificador} & \textbf{Título de la tarea de desarrollo} & \makecell{\textbf{Estimación}\\\textbf{(h)}} \\
                        \hline
                        HT18.1-1 & Configurar build para generar binario único & 1 \\
                        \hline
                        HT18.1-2 & Verificar funcionamiento del binario en distintos entornos & 1 \\
                        \hline
                    \end{tabular}
                    \vspace{0.5em}
                \end{minipage}
            } \\
            \hline
            \multicolumn{3}{|p{\textwidth}|}{
                \textbf{Pruebas de aceptación:}
                    \begin{itemize}
                        \item El binario se genera correctamente y es ejecutable.
                        \item El binario funciona en los sistemas operativos objetivo.
                    \end{itemize}
            }\\
            \hline
            \multicolumn{3}{|p{\textwidth}|}{
                \textbf{Observaciones:}
                \begin{itemize}
                    \item Documentar el proceso de build en el README.
                \end{itemize}
            }\\
            \hline
        \end{tabularx}
    \end{center}
\end{table}

% HT19: Dockerización
\begin{table}[H]
    \begin{center}
        \begin{tabularx}{\textwidth}{|l|X|l|}
            \hline
            \textbf{Identificador HT19} & 
            \textbf{Crear imagen Docker del servidor} &
            \textbf{Estimación: 3}\\
            \hline
            \multicolumn{3}{|c|}{
                \begin{minipage}{\linewidth}
                    \centering
                    \vspace{0.5em}
                    \begin{tabular}{|l|p{8cm}|r|}
                        \hline
                        \textbf{Identificador} & \textbf{Título de la tarea de desarrollo} & \makecell{\textbf{Estimación}\\\textbf{(h)}} \\
                        \hline
                        HT19-1 & Crear Dockerfile para el servidor & 1 \\
                        \hline
                        HT19-2 & Configurar variables de entorno y volúmenes & 1 \\
                        \hline
                        HT19-3 & Probar despliegue local y documentar uso & 1 \\
                        \hline
                    \end{tabular}
                    \vspace{0.5em}
                \end{minipage}
            } \\
            \hline
            \multicolumn{3}{|p{\textwidth}|}{
                \textbf{Pruebas de aceptación:}
                    \begin{itemize}
                        \item El servidor se ejecuta correctamente en un contenedor Docker.
                        \item Se pueden configurar variables y volúmenes.
                    \end{itemize}
            }\\
            \hline
            \multicolumn{3}{|p{\textwidth}|}{
                \textbf{Observaciones:}
                \begin{itemize}
                    \item Seguir buenas prácticas de Docker (multi-stage build si es posible).
                \end{itemize}
            }\\
            \hline
        \end{tabularx}
    \end{center}
\end{table}

% HT20.1: Documentación del proyecto en GitHub
\begin{table}[H]
    \begin{center}
        \begin{tabularx}{\textwidth}{|l|X|l|}
            \hline
            \textbf{Identificador HT20.1} & 
            \textbf{Documentar la instalación y uso del proyecto en el repositorio de Github} &
            \textbf{Estimación: 5}\\
            \hline
            \multicolumn{3}{|c|}{
                \begin{minipage}{\linewidth}
                    \centering
                    \vspace{0.5em}
                    \begin{tabular}{|l|p{8cm}|r|}
                        \hline
                        \textbf{Identificador} & \textbf{Título de la tarea de desarrollo} & \makecell{\textbf{Estimación}\\\textbf{(h)}} \\
                        \hline
                        HT20.1-1 & Redactar README con instrucciones de instalación & 2 \\
                        \hline
                        HT20.1-2 & Documentar configuración y variables de entorno & 1 \\
                        \hline
                        HT20.1-3 & Añadir ejemplos de uso y comandos básicos & 1 \\
                        \hline
                        HT20.1-4 & Revisar y mejorar formato y claridad & 1 \\
                        \hline
                    \end{tabular}
                    \vspace{0.5em}
                \end{minipage}
            } \\
            \hline
            \multicolumn{3}{|p{\textwidth}|}{
                \textbf{Pruebas de aceptación:}
                    \begin{itemize}
                        \item El README permite instalar y ejecutar el proyecto desde cero.
                        \item Toda la configuración necesaria está documentada.
                    \end{itemize}
            }\\
            \hline
            \multicolumn{3}{|p{\textwidth}|}{
                \textbf{Observaciones:}
                \begin{itemize}
                    \item Usar ejemplos claros y comandos reproducibles.
                \end{itemize}
            }\\
            \hline
        \end{tabularx}
    \end{center}
\end{table}

% HT20.2: Documentación de la API REST con OpenAPI
\begin{table}[H]
    \begin{center}
        \begin{tabularx}{\textwidth}{|l|X|l|}
            \hline
            \textbf{Identificador HT20.2} & 
            \textbf{Documentar mediante la generación de una página web todos los endpoints de la API REST} &
            \textbf{Estimación: 5}\\
            \hline
            \multicolumn{3}{|c|}{
                \begin{minipage}{\linewidth}
                    \centering
                    \vspace{0.5em}
                    \begin{tabular}{|l|p{8cm}|r|}
                        \hline
                        \textbf{Identificador} & \textbf{Título de la tarea de desarrollo} & \makecell{\textbf{Estimación}\\\textbf{(h)}} \\
                        \hline
                        HT20.2-1 & Generar especificación OpenAPI de la API REST & 2 \\
                        \hline
                        HT20.2-2 & Añadir descripciones y ejemplos a los endpoints & 1 \\
                        \hline
                        HT20.2-3 & Publicar documentación como página web (Swagger UI u OpenAPI UI) & 1 \\
                        \hline
                        HT20.2-4 & Revisar y mantener la documentación actualizada & 1 \\
                        \hline
                    \end{tabular}
                    \vspace{0.5em}
                \end{minipage}
            } \\
            \hline
            \multicolumn{3}{|p{\textwidth}|}{
                \textbf{Pruebas de aceptación:}
                    \begin{itemize}
                        \item Todos los endpoints están documentados y accesibles vía web.
                        \item La documentación incluye ejemplos de peticiones y respuestas.
                    \end{itemize}
            }\\
            \hline
            \multicolumn{3}{|p{\textwidth}|}{
                \textbf{Observaciones:}
                \begin{itemize}
                    \item Usar herramientas automáticas para mantener la documentación sincronizada.
                \end{itemize}
            }\\
            \hline
        \end{tabularx}
    \end{center}
\end{table}

